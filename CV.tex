\documentclass[11pt]{article}
\usepackage[a4paper, margin=0.5in]{geometry}
\usepackage[spanish]{babel}
\usepackage[utf8]{inputenc}
\usepackage{hyperref}
\usepackage{enumitem}
\usepackage{xcolor}
\hypersetup{
  colorlinks=true,
  linkcolor=black,
  urlcolor=gray
}
\pagestyle{empty}

\begin{document}

\begin{center}
  {\Large \textbf{DAVID ARMANDO RAMÍREZ NAVARRETE}}\\[2pt]
  Desarrollador COBOL
\end{center}

\vspace{0.2cm}

\section*{Perfil}
Ingeniero en Sistemas Computacionales con experiencia en desarrollo de aplicaciones web y entornos mainframe. Apasionado por la programación, con enfoque en soluciones eficientes e innovadoras. Comprometido con el aprendizaje continuo y la mejora de habilidades técnicas, especialmente en desarrollo de software. Adaptable, proactivo y con alta capacidad para trabajar en equipo y adoptar nuevas tecnologías.

\vspace{0.2cm}

\section*{Tecnologías}
\textbf{Lenguajes:} Java, JavaScript, Python, COBOL \\
\textbf{Consulta de Datos:} SQL, QMF \\
\textbf{Control de Versiones:} ChangeMan, Git, Bitbucket \\
\textbf{Metodologías:} Marco de Trabajo Agile, Waterfall \\
\textbf{Plataformas y Otros:} Google Cloud Platform, TSO, ISPF, DB2, JCL, QMF

\vspace{0.2cm}

\section*{Habilidades Blandas}
Trabajo en Equipo, Adaptabilidad, Compromiso, Análisis de problemas, Resolución de problemas, Orientación a resultados, Autodidacta, Gestión del tiempo

\vspace{0.2cm}

\section*{Experiencia Profesional}

\textbf{NTT DATA - BBVA} \\
Becario COBOL \hfill Julio 2025 -- Actualidad \\
\begin{itemize}[leftmargin=*]
  \item Revisión de documentación por cambios productivos post instalación.
  \item Validación de los componentes en el ambiente de producción a nivel documental.
  \item Revisión de casos de prueba y evidencias.
  \item Revisión y actualización de información en la base de datos post instalación.
  \item Creación de tablas dinámicas en la herramienta Looker Studio para generar reportes de productividad para el área.
  \item Administración y control de periodos vacacionales del equipo.
  \item Ejecución de despliegues y reinicios en el entorno de desarrollo.
  \item \textbf{Herramientas Utilizadas:} Google Looker Studio, Google App Script, Jira, Google Sheets (Funciones avanzadas).
\end{itemize}

\vspace{0.2cm}

\textbf{NTT Data - Becario} \\
Programa de Formación COBOL \\ 
\begin{itemize}[leftmargin=*]
  \item Desarrollo y mantenimiento de programas COBOL para procesos bancarios.
  \item Optimización de consultas SQL en entornos mainframe.
  \item Implementación de soluciones en JCL y DB2.
  \item Colaboración en equipos Agile para entrega de proyectos.
\end{itemize}

\vspace{0.2cm}

\textbf{Google Cloud Platform (GCP) - Proyecto Personal} \\
Desarrollador Cloud \\
\begin{itemize}[leftmargin=*]
  \item Configuración de entornos en GCP.
  \item Automatización de tareas con scripts personalizados.
  \item Gestión de máquinas virtuales y recursos en la nube.
  \item Documentación y monitoreo de sistemas.
  \item Entrega de soluciones escalables y trabajo autónomo.
\end{itemize}

\vspace{0.2cm}

\section*{Educación}
\textbf{Universidad Virtual del Estado de Guanajuato} \\
Ingeniería en Sistemas Computacionales

\vspace{0.2cm}

\section*{Certificaciones y Cursos}
\begin{itemize}[leftmargin=*]
  \item Certificación HOST Standard (Desarrollo Seguro)
  \item Scrum Master
  \item Google Cloud Computing Foundations (Google)
  \item Oracle Next Education – G3 (Alura Latam)
\end{itemize}

\vspace{0.2cm}

\section*{Idiomas}
\begin{itemize}
    \item \textbf{Español:} Nativo 
    \item \textbf{Inglés:} Nivel B1
\end{itemize}

\end{document}