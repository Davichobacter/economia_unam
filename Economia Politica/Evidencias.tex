\documentclass[12pt]{article}

% Configuración básica
\usepackage[spanish]{babel}
\usepackage[utf8]{inputenc}
\usepackage[T1]{fontenc}
\usepackage{geometry}
\usepackage{setspace}
\usepackage{hyperref}
\usepackage{enumitem}
\usepackage{graphicx}
\usepackage[section]{placeins} 

\geometry{letterpaper, margin=2.5cm}
\onehalfspacing

% Datos del informe (puedes modificarlos)
\title{Informe Técnico de Hallazgos e Inconsistencias Detectadas}
\author{David Armando Ramírez Navarrete}
\date{01 de diciembre de 2025}

\begin{document}

\maketitle

\section*{Resumen}
El presente informe técnico documenta los hallazgos obtenidos durante la revisión inicial de la documentación de Looker Studio, como parte del análisis de información realizado en las primeras semanas de prácticas. El trabajo refleja habilidades de análisis, revisión estructurada de documentación técnica y atención al detalle.

\section*{Introducción}
Las prácticas profesionales representan una oportunidad para aplicar conocimientos académicos en un entorno real de trabajo. Durante las primeras semanas se desarrolló un proceso de análisis de documentación técnica con el fin de comprender herramientas, flujos de datos y procesos utilizados en la organización.

Este informe presenta los hallazgos obtenidos a partir de la revisión de la documentación oficial de Looker Studio, con énfasis en la comprensión de la plataforma y en la identificación de oportunidades de mejora o aprendizaje relevante para el proyecto asignado.

\section*{Características formativas}

\subsection*{Saber: Análisis de información}
Durante esta fase de la práctica se fortalecieron habilidades relacionadas con:
\begin{itemize}
    \item Interpretación de documentación técnica.
    \item Identificación de información clave para la construcción de reportes.
    \item Comprensión de flujos de datos y estructuras requeridas para análisis visual.
\end{itemize}

\subsection*{Saber hacer: Revisión estructurada de documentación técnica}
Se aplicó una metodología ordenada para:
\begin{itemize}
    \item Revisar documentación oficial de plataformas de análisis de datos.
    \item Analizar ejemplos y guías de uso de Looker Studio.
    \item Evaluar procedimientos para migración, transformación y visualización de datos.
\end{itemize}

\subsection*{Saber ser: Atención al detalle y responsabilidad}
Durante el análisis se mantuvo un enfoque en:
\begin{itemize}
    \item La precisión en la interpretación de conceptos técnicos.
    \item La responsabilidad en el registro y documentación de hallazgos.
    \item La disposición para identificar oportunidades de mejora.
\end{itemize}

\section*{Objetivos de la práctica profesional}

\subsection*{Objetivo general}
Analizar la documentación técnica y los sistemas internos de la organización para identificar hallazgos y oportunidades de mejora, apoyando el desarrollo del proyecto asignado.

\subsection*{Objetivos específicos}
\begin{itemize}
    \item Analizar, durante las primeras seis semanas, la documentación técnica y los sistemas internos para identificar al menos tres oportunidades de mejora y asegurar la comprensión de los procesos del área.
    \item Elaborar reportes y actualizar bases de datos para apoyar el seguimiento de indicadores y la trazabilidad de la información.
    \item Participar en el desarrollo y depuración de componentes de software para fortalecer habilidades de programación y calidad de código.
\end{itemize}

\section*{Metodología de revisión}
La revisión se realizó siguiendo estos pasos:

\begin{enumerate}[label=\arabic*.]
    \item Identificación de documentación relevante en Looker Studio.
    \item Análisis de ejemplos, guías y especificaciones técnicas.
    \item Evaluación de funcionalidades aplicables al proyecto.
    \item Registro de hallazgos con base en criterios de utilidad, aplicabilidad y relevancia técnica.
\end{enumerate}

\section*{Hallazgos principales}
Durante la revisión inicial de la documentación de Looker Studio se identificaron los siguientes hallazgos relevantes para el desarrollo del proyecto:

\subsection*{Hallazgo 1: Migración de reportes desde hojas de cálculo}
Se identificó el procedimiento para trasladar un reporte almacenado en una hoja de cálculo hacia Looker Studio. Esto incluye la conexión con la fuente de datos, la configuración inicial del reporte y la adecuada estructuración de las tablas importadas.  
Este hallazgo facilitará la automatización y actualización dinámica de reportes actualmente generados de forma manual.

\subsection*{Hallazgo 2: Uso de código en campos calculados}
Se revisó la sintaxis necesaria para crear campos calculados mediante expresiones personalizadas. Asimismo, se comprendió cómo integrarlos en diferentes tipos de gráficos para generar métricas más complejas y adaptadas a las necesidades del análisis.  
Este conocimiento permitirá crear indicadores más precisos y visualizaciones más robustas dentro de los reportes.

\subsection*{Hallazgo 3: Generación de texto dinámico mediante variables de entorno}
Se identificó cómo generar texto dinámico dentro de los reportes utilizando variables de entorno y parámetros configurables. Esto permite que los títulos, descripciones y anotaciones cambien automáticamente en función de valores definidos por el usuario o el contexto.  
Este hallazgo contribuye a la creación de reportes interactivos y más flexibles.

\section*{Actividades realizadas}
Las principales actividades realizadas durante este periodo fueron:

\begin{itemize}
    \item Revisión detallada de la documentación técnica de Looker Studio.
    \item Experimentación con fuentes de datos externas para comprender el proceso de importación.
    \item Creación de ejemplos iniciales de campos calculados.
    \item Pruebas con variables de entorno para generar contenido dinámico.
    \item Registro sistemático de hallazgos y oportunidades de mejora.
\end{itemize}

\section*{Evidencias}
A continuación se presentan capturas de pantalla que ilustran los resultados obtenidos durante la revisión y experimentación con Looker Studio.

\begin{figure}[h!]
    \centering
    \includegraphics[width=0.8\textwidth]{Resultado 1.png}
    \caption{Resultado de reporte migrado a tablero de Looker Studio.}
\end{figure}

\begin{figure}[h!]
    \centering
    \includegraphics[width=0.8\textwidth]{Resultado 2.png}
    \caption{Ejemplo de uso de variables de entorno para texto dinámico en Looker Studio.}
\end{figure}

\begin{figure}[h!]
    \centering
    \includegraphics[width=0.8\textwidth]{Resultado 3.png}
    \caption{Ejemplo texto dinámico en Looker Studio.}
\end{figure}

\begin{figure}[h!]
    \centering
    \includegraphics[width=0.8\textwidth]{Resultado 4.png}
    \caption{Ejemplo de reporte con campos calculados y texto dinámico en Looker Studio.}
\end{figure}

\FloatBarrier 

\section*{Conclusiones}
El análisis inicial de Looker Studio permitió comprender las capacidades de la herramienta y sentar las bases para el desarrollo posterior de reportes interactivos y automatizados. Los hallazgos documentados representan avances importantes para mejorar la forma en que se generan, actualizan y presentan los datos dentro del proyecto.

\section*{Referencias}

\begin{itemize}
    \item Google Cloud. (n.d.). \textit{Cómo crear campos calculados en Looker Studio}.  
    Recuperado de: \url{https://docs.cloud.google.com/looker/docs/studio/tutorial-create-calculated-fields-in-looker-studio?hl=es-419}

    \item Google Cloud. (n.d.). \textit{Variables de resultado de consulta en Looker Studio}.  
    Recuperado de: \url{https://docs.cloud.google.com/looker/docs/studio/query-result-variables?hl=es-419}

    \item Google Cloud. (n.d.). \textit{Agregar, editar y solucionar problemas con campos calculados en Looker Studio}.  
    Recuperado de: \url{https://docs.cloud.google.com/looker/docs/studio/add-edit-and-troubleshoot-calculated-fields?hl=es-419}
\end{itemize}

\end{document}

