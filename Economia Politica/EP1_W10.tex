\documentclass[stu,12pt,floatsintext]{apa7}


\usepackage[american]{babel}
\usepackage{csquotes}
\usepackage[style=apa,sortcites=true,sorting=nyt,backend=biber]{biblatex}
\DeclareLanguageMapping{american}{american-apa} % APA (7.ª con biblatex-apa actual)
\addbibresource{Bib.bib} % Bibliography file


\DefineBibliographyStrings{american}{
  references = {Referencias},
}

\usepackage[T1]{fontenc} 
\usepackage{mathptmx} 

% Title page stuff _____________________
\title{Semana 10 - La  Contribución Teórica de David Ricardo} % The big, long version of the title for the title page
% \shorttitle{APA Starter} % The short title for the header
\author{David Armando Ramirez Navarrete}
\duedate{19 octubre 2025}
% \date{January 17, 2024} The student version doesn't use the \date command, for whatever reason
\affiliation{Facultad de Economía (UNAM)}
\course{Economía Política I} % LaTeX gets annoyed (i.e., throws a grumble-error) if this is blank, so I put something here. However, if your instructor will mark you off for this being on the title page, you can leave this entry blank (delete the PSY 4321, but leave the command), and just make peace with the error that will happen. It won't break the document.
\professor{Gabriela Téllez Arriaga}  % Same situation as for the course info. Some instructors want this, some absolutely don't and will take off points. So do what you gotta.

% \abstract{This is the abstract for this paper, wherein the main points of the introduction, method, results, and discussion are quickly talked about. Probably in more than one sentence, though. Dare I guess, more than two? There is a page break before starting the Introduction.}

%\keywords{APA style, demonstration} % If you need to have keywords for your paper, delete the % at the start of this line

\begin{document}

\maketitle

\section*{Introducción}

\subsection{Datos Bibliográficos}

Capítulo IV, "David Ricardo", contenido en la obra Fisiocracia, Smith, Ricardo y Marx, escrito por Claudio Napoleoni. La edición consultada fue publicada por Oikos -Tau en España en 1981, abarcando las páginas 67 a 100. Intelectual y disciplinariamente, el texto se sitúa en el ámbito de la economía política clásica, proporcionando un análisis crítico y comparativo del pensamiento ricardiano, contrastándolo con sus predecesores (los fisiócratas y Adam Smith) y señalando los problemas fundamentales que serían abordados y superados por Karl Marx.

\subsection{Contexto Histórico, Intelectual y Disciplinar}

La idea central de David Ricardo, que define su aproximación a la economía política, es que esta ciencia debe ocuparse primordialmente de la distribución del producto social entre las clases fundamentales de la sociedad: salarios (trabajadores), beneficios (capitalistas) y renta (propietarios de tierras). A diferencia de Smith, quien se enfocaba en la generación de riqueza nacional, Ricardo se concentra en la determinación de las leyes que rigen esta distribución. Su investigación científica se basa en un conocimiento riguroso del carácter capitalista de la economía, superando las consideraciones residuales de la simple sociedad mercantil presentes aún en Smith. Específicamente, el principal problema que aborda Ricardo es la determinación del valor y la evolución del tipo del beneficio (tasa de ganancia), y sus relaciones con el tipo del salario.

\subsection{Propósito o Tesis}

Se infiere que el \textbf{problema principal} que aborda es el \textbf{análisis profundo de la contribución teórica de David Ricardo}, examinando cómo sus ideas se relacionan con las de sus predecesores (Fisiocracia y Smith) y cómo influyen en el desarrollo posterior del pensamiento económico (Marx). La tesis central del capítulo se enfoca en destacar los \textbf{aportes estructurales de Ricardo a la teoría económica}, en áreas como la teoría del valor-trabajo, la renta diferencial, o la distribución de la riqueza entre las clases sociales.

\section*{Desarrollo}

\subsection{El Análisis Ricardiano de la Distribución y el Tipo del Beneficio}

La economía capitalista, según Ricardo, está estructurada en tres clases (trabajadores, propietarios del capital y propietarios de tierras), y el desarrollo económico solo puede investigarse rigurosamente a partir de la consideración de cómo estas clases participan en el producto social. La parte del producto social asignada a cada clase se relaciona con los recursos que posee: salarios con el trabajo empleado, renta con la tierra y beneficios con el capital invertido, lo que da lugar al tipo del salario y al tipo del beneficio, siendo este último la magnitud económica fundamental que impulsa el proceso capitalista.
Inicialmente, en su Ensayo sobre la influencia de un bajo precio en el grano (1815), Ricardo abordó la cuestión del tipo del beneficio basándose en la evolución de la renta territorial.

\begin{enumerate}
  \item \textbf{Teoría de la Renta Diferencial y la Caída del Beneficio Agrícola}: La renta diferencial surge cuando la demanda exige el cultivo de tierras sucesivamente menos fértiles o peor situadas (tierras marginales). La tierra marginal no paga renta. A medida que el cultivo se expande, la cantidad de capital (incluidos los salarios) necesaria para obtener la misma cantidad de producto aumenta. Puesto que la renta se determina por la diferencia de productividad respecto a la tierra marginal, el tipo del beneficio en la agricultura (que se forma en la tierra marginal) tiende a caer. El carácter determinante de la agricultura implicaba que la competencia transmitiría esta tendencia descendente al tipo general del beneficio en toda la economía (\cite{napoleoni1981}).
  \item \textbf{El Problema de los Precios y la Necesidad de una Teoría del Valor}: Posteriormente, Ricardo reconoció que el cálculo del beneficio en términos puramente físicos (la hipótesis del "grano", donde producto y capital son idénticos) resultaba insostenible, pues la expansión del cultivo provoca un aumento en el precio relativo del grano respecto a otras mercancías. Este aumento del precio del grano eleva el costo del capital anticipado en salarios, lo que, en el sector industrial, provoca una disminución del tipo del beneficio (\cite{napoleoni1981}). La caída del beneficio industrial, a través de la competencia, se transmite al sector agrícola, regulando el tipo general de beneficio. Esta constatación forzó a Ricardo a concentrar su atención en el problema de los precios y, consecuentemente, a formular una teoría del valor general en sus Principios (1817).
\end{enumerate}

\subsection{Teoría del Valor-Trabajo}

Ricardo adoptó la Teoría del Valor-Trabajo contenido, criticando la postura de Smith del "trabajo demandable". Ricardo argumentó que definir el valor de cambio en función del trabajo que la mercancía puede "poner en movimiento" resultaba en un círculo vicioso, ya que presupone la relación de cambio que se intenta determinar.

\begin{enumerate}
  \item \textbf{Labor Contenido y Crítica a Smith}: Ricardo sostuvo que, al igual que en las economías precapitalistas, el valor de cambio de las mercancías en la economía capitalista es determinado primariamente por la cantidad de trabajo requerido para su producción, incluyendo el trabajo directo y el contenido indirectamente en los medios de producción (\cite{napoleoni1981}).
  
  \begin{itemize}
    \item Nota Marxista: La fuente utiliza nociones marxistas para reinterpretar la crítica de Ricardo a Smith. Según la formulación marxista, el cambio real ocurre entre capital y fuerza de trabajo, la cual es una mercancía cuyo valor está determinado por el trabajo necesario para la subsistencia (salario). Ricardo acierta al señalar que la ley general del cambio (valor por trabajo contenido) se mantiene en este intercambio. No obstante, al centrarse solo en este cambio "real", Ricardo pierde de vista la generación de la plusvalía (el plustrabajo no pagado, que es la fuente del beneficio y la renta), un aspecto al que Smith, de manera confusa, hacía referencia con su concepto de trabajo demandable.
  \end{itemize}
  
  \item \textbf{Las Dificultades de la Teoría del Valor-Trabajo}: El argumento de que los valores de cambio son iguales a las relaciones de trabajo contenido le permitiría a Ricardo determinar el tipo del beneficio en términos físicos (análogos al esquema del grano, pero con el trabajo como unidad). Sin embargo, el autor encuentra que el trabajo contenido es incompatible con la realidad del mercado competitivo (\cite{napoleoni1981}).
  
    \begin{itemize}
      \item Estructura Temporal del Capital: La formación de un tipo general de beneficio, resultado de la competencia, implica que la relación de cambio entre mercancías no depende solo de la cantidad total de trabajo contenido, sino también de la estructura temporal de la inversión de ese trabajo (la proporción de capital fijo o la antigüedad de la inversión). Mercancías con una cantidad idéntica de trabajo contenido, pero con diferentes estructuras temporales (mayor inversión en épocas más lejanas), tendrán valores relativos diferentes para asegurar el tipo general de beneficio.
      \item La Medida Invariable del Valor: Esta dificultad se manifiesta en la imposibilidad de encontrar una “perfecta unidad de medida” invariable para el valor. Un valor absoluto (definido como valor de cambio referido a una unidad de medida invariable de trabajo contenido) fracasa porque la relación entre el valor de esta unidad y los valores de otras mercancías varía con los cambios en la distribución (el tipo general de beneficio), incluso si las cantidades de trabajo no cambian.
    \end{itemize}
    
  \item \textbf{El Resultado Aproximado}: Ante estas dificultades, Ricardo se conformó con una determinación aproximada del valor, considerando que la cantidad de trabajo contenido era el elemento más importante, aunque no el único, en la determinación del valor. A pesar de los fallos analíticos en su teoría del valor, Ricardo mantuvo su tesis original de que la disminución tendencial del tipo del beneficio agrícola impulsa la caída del beneficio general. Para mantener esta tesis, sin embargo, debe apoyarse en hipótesis simplificadoras, tales como la suposición de que el grano es la mercancía principal en la subsistencia del trabajador y que la productividad del trabajo agrícola es decreciente, lo que hace que la cantidad de trabajo necesaria para producir el salario aumente. Estas hipótesis, sin embargo, hacen que la teoría general del valor sea superflua para la demostración de la caída del beneficio, ya que el tipo de beneficio podría calcularse en términos de grano sin recurrir a los valores (\cite{napoleoni1981}).
\end{enumerate}

\subsection{El Conflicto de Clases y la Maquinaria}
Ricardo logró precisar los términos del conflicto de clases entre propietarios de tierras y la burguesía. Si los productos agrícolas constituyen la mayor parte de los bienes consumidos por los asalariados, cualquier aumento en el precio de estos productos incrementa la renta (favoreciendo a los propietarios) y simultáneamente reduce el tipo del beneficio (al aumentar el coste del salario para el capitalista). El interés capitalista radica en reducir los precios agrícolas, apoyando medidas como la importación de grano, mientras que los propietarios se oponen (\cite{napoleoni1981}).

En cuanto a la relación entre burguesía y proletariado, Ricardo consideraba que el salario tendería al nivel de subsistencia, haciendo que la posición real del obrero fuese neutral en el conflicto de distribución. El conflicto, para Ricardo, se origina en la cuestión de la ocupación.

\begin{enumerate}
  \item \textbf{\emph{Ingreso Bruto vs. Neto}}: Ricardo ponía el foco en el producto neto (que genera beneficios y permite sostener el trabajo improductivo, como ejércitos), minimizando la importancia del producto bruto (del cual depende la capacidad de ocupar trabajo). Esto refleja la tendencia capitalista a buscar el beneficio como fin en sí mismo.
  \item \textbf{\emph{Maquinaria y Desempleo}}: En la tercera edición de sus Principios, Ricardo modificó su postura inicial, argumentando que la introducción de maquinaria, al reorientar el capital (de circulante a fijo), podía reducir de manera permanente el nivel de la ocupación (producto bruto), incluso si el producto neto (beneficio) se mantenía igual o aumentaba. Este argumento evidenció un posible conflicto de fondo entre burguesía y proletariado, aunque Ricardo lo consideró posible, no inevitable, sugiriendo que la acumulación de un mayor beneficio podría reinsertar a los trabajadores desplazados.
\end{enumerate}

\subsection{La Negación de las Crisis de Sobreproducción (Ricardo vs. Malthus)}
Ricardo se opuso persistentemente a la posibilidad de crisis de superproducción general, al considerar que la producción siempre crea su propia demanda.

En contraste, Malthus argumentó que la demanda efectiva podría ser insuficiente para que las mercancías se vendieran a su valor real (es decir, cubriendo también el beneficio al tipo corriente), ya que los capitalistas, al estar inclinados a ahorrar, podrían no proporcionar la demanda necesaria más allá del consumo de los trabajadores productivos. Malthus concluyó que el equilibrio capitalista requería la existencia de una cantidad suficiente de consumidores improductivos (terratenientes, servidores, empleados públicos) que introdujeran en el mercado la demanda efectiva necesaria para realizar el beneficio.

Ricardo refutó el modelo de Malthus mediante un esquema que demostraba cómo el ingreso, al convertirse en capital (mantenimiento de trabajadores productivos adicionales), siempre aseguraría la demanda, permitiendo que el proceso de acumulación prosiguiera hasta el infinito sin generar deficiencias (\cite{napoleoni1981}).

\section*{Diferencias y similitudes entre Adam Smith y David Ricardo}

\subsection{Diferencias en la Definición y el Enfoque de la Economía}

Una de las diferencias más marcadas radica en cómo definieron el objeto de la ciencia económica:

\begin{itemize}
  \item \textbf{Definición de Economía}: Smith definió la economía como la ciencia de la riqueza de las naciones, ocupándose de los medios para maximizar la riqueza . En contraste, Ricardo definió la economía política como la ciencia que se ocupa de la distribución del producto social entre las clases en las que la sociedad se halla dividida: salarios, beneficios y rentas. Para Ricardo, la determinación de las leyes que rigen esta distribución es el problema primordial de la Economía Política (\cite{napoleoni1981}).
  \item \textbf{Conocimiento del Capitalismo}: Ricardo logró un progreso claro respecto a Smith en el conocimiento del carácter capitalista de la economía. Aunque Smith considera el carácter capitalista de la economía, todavía no tiene plena fe en esta consideración y su obra contiene "residuos" de temas de la simple sociedad mercantil (productores independientes). En la obra de Ricardo, estos residuos desaparecen casi por completo; la economía que considera es rigurosamente capitalista, dividida en las tres clases de trabajadores, propietarios del capital y propietarios de tierras (\cite{napoleoni1981}).
\end{itemize}

\subsection{Similitudes y Contrastes en la Teoría del Valor}

Ricardo se movió en el ámbito de un precedente \textit{smithiano} para elaborar su teoría del valor, aunque adoptó una postura crítica.

\begin{itemize}
  \item \textbf{El Problema del Trabajo como Medida/Causa}: Smith buscó una "medida real" del valor de cambio, sugiriendo el trabajo economizado (o "trabajo demandable"), ya que las cantidades iguales de trabajo son siempre de igual valor para el trabajador.
  
  \begin{itemize}
    \item Ricardo critica esta posición, señalando que la determinación de la cantidad de trabajo que una mercancía puede "poner en movimiento" (trabajo demandable) requiere la previa determinación de la relación de cambio, lo que genera un círculo vicioso (\cite{napoleoni1981}).
    \item Smith, al introducir el capital y la propiedad territorial, abandonó la teoría del valor-trabajo (trabajo contenido), sustituyéndola por la "teoría de la suma" de los ingresos (salarios, beneficios y rentas). Ricardo, por su parte, considera que Smith limita la aplicación del principio de la cantidad de trabajo contenida al "estado primitivo y rudo de la sociedad" (\cite{dobb1998}).
  \end{itemize}
  
  \item \textbf{Trabajo Contenido vs. Trabajo Demandable}: Ricardo interpreta que el hecho de que en la economía capitalista parte del producto se transforme en beneficio o renta no impide que las mercancías se cambien según los trabajos en ellas contenidos.
  
  \begin{itemize}
    \item La crítica ricardiana argumenta que, si se mantiene constante la definición de trabajo contenido, la ley del valor-trabajo debería aplicarse tanto en la economía precapitalista como en la capitalista (\cite{napoleoni1981}).
    \item A pesar de su defensa de la ley del valor-trabajo, Ricardo encontró que esta ley se modificaba en la realidad del mercado competitivo. Concluyó que los valores relativos dependen no solo de la cantidad de trabajo contenido, sino también de las estructuras temporales en que se invierte el trabajo (la duración o composición del capital).
    \item Ricardo acusó a Smith de confundir el "trabajo incorporado" con el "trabajo comandado" (\cite{dobb1998}).
  \end{itemize}

\end{itemize}

\subsection{Diferencias en la Distribución, la Renta y el Beneficio}

Ricardo situó la determinación del valor y la evolución del tipo del beneficio en el centro de su teoría económica.

\begin{itemize}
  \item \textbf{Origen de la Renta y el Beneficio}: Smith esbozó una "teoría de la deducción" al tratar el beneficio y la renta como deducciones de lo que "naturalmente" u "originariamente" es producto del trabajo. Ricardo también se centró en la plusvalía (aunque no la conceptualizó como Marx).
  
  \item \textbf{Determinación del Beneficio}: Ricardo abordó la cuestión del tipo del beneficio basándose en la evolución de la renta territorial, determinada por la necesidad de cultivar tierras cada vez menos fértiles. La tendencia a la disminución del tipo del beneficio agrícola se transmite al tipo general del beneficio en el sistema debido a la competencia (\cite{napoleoni1981}).

  \begin{itemize}
    \item Ricardo criticó las inadecuadas explicaciones de Smith sobre la tendencia decreciente del beneficio basada en la competencia general.
  \end{itemize}

  \item \textbf{Renta de la Tierra}: Smith consideró que la renta entra en la composición del precio de las mercancías de forma distinta a los salarios y el beneficio, ya que la renta alta o baja es el efecto del precio, no la causa, y es un precio de monopolio (\cite{dobb1998}).
  
  \item \textbf{Definición de Ingreso Neto}
  
  \begin{itemize}
    \item Smith definió el ingreso neto como el fondo de consumo potencial de capitalistas y asalariados (es decir, el producto bruto menos el capital fijo y circulante necesario para mantenerlo intacto).
    \item Ricardo definió el ingreso neto como el excedente (beneficio y renta) después de pagar los salarios, y criticó a Smith por magnificar las ventajas del producto bruto sobre el producto neto.
  \end{itemize}

  \item \textbf{Armonía vs. Conflicto de Clases}: Smith vio una estricta e inseparable vinculación entre el interés de los terratenientes y los asalariados con el interés general de la sociedad, mientras que advertía de los intereses de los comerciantes y dueños de manufacturas, que buscan el monopolio (\cite{dobb1998}).
  
  \begin{itemize}
    \item Ricardo precisó los términos del conflicto de clases entre propietarios de tierras y la burguesía, particularmente en relación con el precio de los productos agrícolas (como el grano y la importación). También vislumbró el posible conflicto entre burguesía y proletariado respecto a la ocupación y la introducción de máquinas.
  \end{itemize}

\end{itemize}

\subsection{Similitudes Clave}

A pesar de sus diferencias analíticas, Smith y Ricardo compartieron fundamentos esenciales de la escuela clásica:

\begin{itemize}
  \item \textbf{Orden Natural y Leyes Objetivas}: Ambos compartieron la concepción de un sistema económico impulsado por un ímpetu propio y regido por leyes económicas específicas y objetivas. Para Smith, esto se evidenciaba en la "mano invisible" y la noción de que la libre interacción individual produce un "modelo metódico que está lógicamente determinado".
  \item \textbf{Capital como Anticipo (Adelanto)}: Ambos autores adoptaron la noción de capital como "adelanto" (anticipo), especialmente de salarios a los trabajadores, lo cual fue un concepto central de la Economía Política Clásica en Inglaterra.
  \item \textbf{Competencia y Precios Naturales}: Smith postuló que la competencia establece los "valores naturales" (precios de equilibrio), hacia los cuales gravitan continuamente los precios de mercado. Este funcionamiento de la oferta y la demanda para igualar el precio de mercado con el precio natural asegura que la cantidad empleada anualmente se adecue a la demanda efectiva. Ricardo también se basó en el proceso competitivo para explicar la igualdad de los tipos de beneficio en todos los sectores.
  \item \textbf{Acumulación de Capital}: Ambos reconocieron el papel central de la acumulación de capital como motor del progreso económico y el bienestar social, aunque Smith la vinculó directamente con el aumento de la demanda de trabajo.
\end{itemize}

\section*{Conclusión}

El pensamiento económico de David Ricardo representó un punto de inflexión decisivo dentro de la economía política clásica. Su enfoque riguroso sobre las leyes que determinan la distribución del producto social permitió dotar a la disciplina de un carácter más científico y estructurado. Al desplazar el centro de atención desde la mera generación de riqueza, como proponía Smith, hacia la distribución entre las clases sociales, Ricardo inauguró una forma de análisis que anticipó la crítica marxista y las posteriores teorías del valor y la acumulación.

La teoría del valor-trabajo, la renta diferencial y la tendencia decreciente del beneficio constituyen los pilares de su contribución, aun cuando su formulación enfrentó limitaciones analíticas que él mismo reconoció. Su esfuerzo por explicar la dinámica capitalista en términos de leyes objetivas consolidó la idea de que la economía es un sistema de relaciones sociales en movimiento, marcado por tensiones estructurales entre clases y por la búsqueda permanente de equilibrio en la distribución.

En suma, Ricardo elevó el estudio de la economía política a un nivel de abstracción y coherencia que permitió comprender las raíces del desarrollo capitalista y sus contradicciones internas. Su legado, retomado y transformado por Marx, sigue siendo una referencia ineludible para quienes buscan entender la lógica de la producción y la distribución en las sociedades modernas.

\printbibliography

\end{document}

%% 
%% Copyright (C) 2019 by Daniel A. Weiss <daniel.weiss.led at gmail.com>
%% 
%% This work may be distributed and/or modified under the
%% conditions of the LaTeX Project Public License (LPPL), either
%% version 1.3c of this license or (at your option) any later
%% version.  The latest version of this license is in the file:
%% 
%% http://www.latex-project.org/lppl.txt
%% 
%% Users may freely modify these files without permission, as long as the
%% copyright line and this statement are maintained intact.
%% 
%% This work is not endorsed by, affiliated with, or probably even known
%% by, the American Psychological Association.
%% 
%% This work is "maintained" (as per LPPL maintenance status) by
%% Daniel A. Weiss.
%% 
%% This work consists of the file  apa7.dtx
%% and the derived files           apa7.ins,
%%                                 apa7.cls,
%%                                 apa7.pdf,
%%                                 README,
%%                                 APA7american.txt,
%%                                 APA7british.txt,
%%                                 APA7dutch.txt,
%%                                 APA7english.txt,
%%                                 APA7german.txt,
%%                                 APA7ngerman.txt,
%%                                 APA7greek.txt,
%%                                 APA7czech.txt,
%%                                 APA7turkish.txt,
%%                                 APA7endfloat.cfg,
%%                                 Figure1.pdf,
%%                                 shortsample.tex,
%%                                 longsample.tex, and
%%                                 bibliography.bib.
%% 
%%
%%
%% This is file `./samples/shortsample.tex',
%% generated with the docstrip utility.
%%
%% The original source files were:
%%
%% apa7.dtx  (with options: `shortsample')
%% ----------------------------------------------------------------------
%% 
%% apa7 - A LaTeX class for formatting documents in compliance with the
%% American Psychological Association's Publication Manual, 7th edition
%% 
%% Copyright (C) 2019 by Daniel A. Weiss <daniel.weiss.led at gmail.com>
%% 
%% This work may be distributed and/or modified under the
%% conditions of the LaTeX Project Public License (LPPL), either
%% version 1.3c of this license or (at your option) any later
%% version.  The latest version of this license is in the file:
%% 
%% http://www.latex-project.org/lppl.txt
%% 
%% Users may freely modify these files without permission, as long as the
%% copyright line and this statement are maintained intact.
%% 
%% This work is not endorsed by, affiliated with, or probably even known
%% by, the American Psychological Association.
%% 
%% ----------------------------------------------------------------------