\documentclass[stu,12pt,floatsintext]{apa7}


\usepackage[american]{babel}
\usepackage{csquotes}
\usepackage[style=apa,sortcites=true,sorting=nyt,backend=biber]{biblatex}
\DeclareLanguageMapping{american}{american-apa} % APA (7.ª con biblatex-apa actual)
\addbibresource{Bib.bib} % Bibliography file


\DefineBibliographyStrings{american}{
  references = {Referencias},
}

\usepackage[T1]{fontenc} 
\usepackage{mathptmx} 

% Title page stuff _____________________
\title{Semana 7 - Los fisiócratas} % The big, long version of the title for the title page
% \shorttitle{APA Starter} % The short title for the header
\author{David Armando Ramirez Navarrete}
\duedate{28 septiembre 2025}
% \date{January 17, 2024} The student version doesn't use the \date command, for whatever reason
\affiliation{Facultad de Economía (UNAM)}
\course{Economía Política I} % LaTeX gets annoyed (i.e., throws a grumble-error) if this is blank, so I put something here. However, if your instructor will mark you off for this being on the title page, you can leave this entry blank (delete the PSY 4321, but leave the command), and just make peace with the error that will happen. It won't break the document.
\professor{Gabriela Téllez Arriaga}  % Same situation as for the course info. Some instructors want this, some absolutely don't and will take off points. So do what you gotta.

% \abstract{This is the abstract for this paper, wherein the main points of the introduction, method, results, and discussion are quickly talked about. Probably in more than one sentence, though. Dare I guess, more than two? There is a page break before starting the Introduction.}

%\keywords{APA style, demonstration} % If you need to have keywords for your paper, delete the % at the start of this line

\begin{document}

\maketitle

\section*{Introducción}

El estudio de las corrientes precursoras de la economía política clásica revela la importancia fundacional de la fisiocracia, que a mediados del siglo XVIII planteó por primera vez la comprensión del sistema económico como un organismo regido por leyes necesarias. François Quesnay y Victor de Riqueti, marqués de Mirabeau, se convirtieron en sus exponentes principales. Claudio Napoleoni, en su obra Fisiocracia, Smith, Ricardo, Marx, analiza críticamente este legado y subraya la relevancia de los fisiócratas en tanto fueron los primeros en localizar en el proceso productivo el origen del excedente, el denominado produit net, en contraste con tradiciones previas que lo situaban en el intercambio. Aunque limitada por su énfasis exclusivo en la agricultura y por la ausencia de una teoría del valor, la fisiocracia representa un punto de partida necesario para los desarrollos posteriores de la escuela clásica, especialmente en la obra de Adam Smith.

\section*{¿Cuál es el enfoque de la corriente de los fisiócratas?}

El núcleo del pensamiento fisiócrata es la idea de un “orden natural” que rige la vida social del mismo modo en que lo hacen las leyes de la naturaleza física. No obstante, esta analogía tiene un límite: mientras el orden físico es independiente de la voluntad humana, el orden social existe solo en la medida en que los hombres lo aceptan y no lo obstruyen. Como señala \textcite{napoleoni1981}, por lo tanto, el óptimo social, porque asegura beneficios que de otra manera serían inalcanzables.

La base de este orden es económica: la mercantilización general de los productos, que transforma a toda la sociedad en una unidad integrada por medio del intercambio. Este planteamiento respondía a la realidad francesa prerrevolucionaria, predominantemente agrícola, donde la agricultura capitalista de los arrendatarios burgueses había demostrado ser más eficiente que las formas campesinas. De allí que los fisiócratas defendieran la conducción capitalista del campo como la forma más avanzada del sector, aunque restringieron el fenómeno del excedente únicamente a la agricultura, asumiendo una organización artesanal para las manufacturas.

\section*{¿Qué es el excedente para los fisiócratas?}

El produit net designa la parte de la riqueza que excede el consumo necesario del proceso productivo. Su importancia radica en que constituye la base del consumo ampliado y de la inversión, permitiendo el crecimiento económico.

Los fisiócratas lo atribuyeron exclusivamente a la agricultura, puesto que la fertilidad de la tierra era considerada la fuente determinante de su generación (\cite{napoleoni1981}). En consecuencia, solo el trabajo agrícola era definido como productivo, no por su naturaleza intrínseca, sino por su capacidad de aprovechar esa fertilidad.

En términos de valoración, el excedente no podía calcularse en magnitudes físicas homogéneas, por lo que debía expresarse en valores de mercado, lo cual socavaba la pretendida supremacía de la agricultura, ya que exigía considerar las relaciones de intercambio con la industria.

Respecto a la atribución del excedente, Quesnay y Mirabeau sostuvieron que este se resolvía enteramente en la renta territorial percibida por los propietarios de tierras (\cite{napoleoni1981}). A diferencia de los clásicos posteriores, no reconocieron el beneficio como una forma específica del excedente, incluso en la agricultura capitalista. Cualquier ganancia del arrendatario se concebía como transitoria, absorbida por la renta en la renovación de contratos.

\section*{Menciona las clases sociales como Quesnay organiza el sistema económico, en el Tableau économiqué.
Explica las características de cada una de las clases.}

El célebre Tableau économique de Quesnay fue el primer intento de representar el equilibrio global del sistema económico. Allí se expone cómo la riqueza social se distribuye entre clases y cómo se reproducen las condiciones para reiniciar la producción en la misma escala.

El sistema se organiza en tres clases:

\begin{itemize}
\item{La Clase Productiva, integrada por arrendatarios y trabajadores agrícolas, considerada la única capaz de generar excedente. Su rol es proveer alimentos y materias primas, reintegrar su capital y garantizar la continuidad del ciclo productivo.}

\item{La Clase Estéril, formada por artesanos y manufactureros, cuya actividad transforma bienes sin crear excedente. En el Tableau, su función es intercambiar manufacturas por productos agrícolas para sostener el ciclo.}

\item{La Clase de los Propietarios de Tierras, que recibe la totalidad del excedente bajo la forma de renta y lo gasta en alimentos y manufacturas, poniendo en movimiento la circulación de la riqueza. Una renta amplia, según los fisiócratas, permite expandir el proceso productivo mediante inversiones en la tierra (avances foncières).
}
\end{itemize}

El Tableau pretendía mostrar el ordre naturel: la situación donde el produit net se maximiza y determina el alcance del proceso económico.

\section*{Conclusión}

El esquema fisiocrático representa un hito intelectual por situar el origen del excedente en la producción, anticipando el enfoque clásico y marxista (\cite{napoleoni1981}). Su mérito radica en haber concebido la economía como un todo coherente y en haber planteado la reproducción ampliada de la sociedad.

Sin embargo, como advierte \textcite{napoleoni1981}, la fisiocracia muestra limitaciones estructurales: la falta de una teoría del valor que permita medir el excedente, la dependencia de precios empíricos y la restricción del excedente a la agricultura. Además, la negativa a reconocer el beneficio capitalista como parte del excedente redujo el alcance de su análisis.

De este modo, la fisiocracia fue una etapa de transición: su noción de produit net y el Tableau économique marcaron un camino, pero también señalaron los problemas que Smith, Ricardo y posteriormente Marx intentarían resolver. El ideal fisiócrata del ordre naturel, pensado para maximizar la riqueza, se convirtió en un paradigma histórico cuyo valor principal consistió en impulsar el desarrollo de una economía política más rigurosa y generalizable.

\printbibliography

\end{document}

%% 
%% Copyright (C) 2019 by Daniel A. Weiss <daniel.weiss.led at gmail.com>
%% 
%% This work may be distributed and/or modified under the
%% conditions of the LaTeX Project Public License (LPPL), either
%% version 1.3c of this license or (at your option) any later
%% version.  The latest version of this license is in the file:
%% 
%% http://www.latex-project.org/lppl.txt
%% 
%% Users may freely modify these files without permission, as long as the
%% copyright line and this statement are maintained intact.
%% 
%% This work is not endorsed by, affiliated with, or probably even known
%% by, the American Psychological Association.
%% 
%% This work is "maintained" (as per LPPL maintenance status) by
%% Daniel A. Weiss.
%% 
%% This work consists of the file  apa7.dtx
%% and the derived files           apa7.ins,
%%                                 apa7.cls,
%%                                 apa7.pdf,
%%                                 README,
%%                                 APA7american.txt,
%%                                 APA7british.txt,
%%                                 APA7dutch.txt,
%%                                 APA7english.txt,
%%                                 APA7german.txt,
%%                                 APA7ngerman.txt,
%%                                 APA7greek.txt,
%%                                 APA7czech.txt,
%%                                 APA7turkish.txt,
%%                                 APA7endfloat.cfg,
%%                                 Figure1.pdf,
%%                                 shortsample.tex,
%%                                 longsample.tex, and
%%                                 bibliography.bib.
%% 
%%
%%
%% This is file `./samples/shortsample.tex',
%% generated with the docstrip utility.
%%
%% The original source files were:
%%
%% apa7.dtx  (with options: `shortsample')
%% ----------------------------------------------------------------------
%% 
%% apa7 - A LaTeX class for formatting documents in compliance with the
%% American Psychological Association's Publication Manual, 7th edition
%% 
%% Copyright (C) 2019 by Daniel A. Weiss <daniel.weiss.led at gmail.com>
%% 
%% This work may be distributed and/or modified under the
%% conditions of the LaTeX Project Public License (LPPL), either
%% version 1.3c of this license or (at your option) any later
%% version.  The latest version of this license is in the file:
%% 
%% http://www.latex-project.org/lppl.txt
%% 
%% Users may freely modify these files without permission, as long as the
%% copyright line and this statement are maintained intact.
%% 
%% This work is not endorsed by, affiliated with, or probably even known
%% by, the American Psychological Association.
%% 
%% ----------------------------------------------------------------------