\documentclass[stu,12pt,floatsintext]{apa7}


\usepackage[american]{babel}
\usepackage{csquotes}
\usepackage[style=apa,sortcites=true,sorting=nyt,backend=biber]{biblatex}
\DeclareLanguageMapping{american}{american-apa} % APA (7.ª con biblatex-apa actual)
\addbibresource{Bib.bib} % Bibliography file


\DefineBibliographyStrings{american}{
  references = {Referencias},
}

\usepackage[T1]{fontenc} 
\usepackage{mathptmx} 

% Title page stuff _____________________
\title{Semana 9 - La Fundamentación de la Economía Política Clásica} % The big, long version of the title for the title page
% \shorttitle{APA Starter} % The short title for the header
\author{David Armando Ramirez Navarrete}
\duedate{12 octubre 2025}
% \date{January 17, 2024} The student version doesn't use the \date command, for whatever reason
\affiliation{Facultad de Economía (UNAM)}
\course{Economía Política I} % LaTeX gets annoyed (i.e., throws a grumble-error) if this is blank, so I put something here. However, if your instructor will mark you off for this being on the title page, you can leave this entry blank (delete the PSY 4321, but leave the command), and just make peace with the error that will happen. It won't break the document.
\professor{Gabriela Téllez Arriaga}  % Same situation as for the course info. Some instructors want this, some absolutely don't and will take off points. So do what you gotta.

% \abstract{This is the abstract for this paper, wherein the main points of the introduction, method, results, and discussion are quickly talked about. Probably in more than one sentence, though. Dare I guess, more than two? There is a page break before starting the Introduction.}

%\keywords{APA style, demonstration} % If you need to have keywords for your paper, delete the % at the start of this line

\begin{document}

\maketitle

\section*{Introducción}

Maurice Dobb, destacado economista británico adscrito a la tradición marxista, aborda la obra de los fundadores de la economía política con un enfoque centrado en las teorías del valor y la distribución, esenciales para entender la estructura de la explotación capitalista. Desde esta óptica crítica, Dobb analiza la contribución de Adam Smith, figura canónica cuya obra La riqueza de las naciones estableció la economía política clásica. La relevancia de Smith radica en que fue el primero en postular una concepción general de un sistema económico impulsado por su propio ímpetu y regido por leyes económicas específicas. El aporte crucial de Smith fue reemplazar el pensamiento antiguo del “derecho natural” con una afirmación empírica: la libre interacción de los individuos produce un “modelo metódico que está lógicamente determinado” y que puede ser dilucidado en términos racionales (\cite{dobb1998}).

\section*{El Mecanismo Smithiano: Trabajo, Valor y la ``Mano Invisible''}

La principal preocupación subyacente de Smith, según \textcite{dobb1998}, fue la noción del provecho individual como fuerza conductora de la economía, en la que el interés propio impulsa el bienestar general. Dobb señala que, aunque Smith rechazó explícitamente la formulación de Mandeville sobre los “vicios privados” y las “virtudes públicas”, su teoría de la mano invisible cumplió una función similar, mostrando cómo la búsqueda del interés individual podía armonizarse con el interés social. Este mecanismo demostraba la existencia de un orden automático burgués, incompatible con la injerencia del soberano o del estadista, y se erigía como la innovación central de Smith en el pensamiento económico.

Dentro de este sistema autorregulador, la competencia desempeña un papel crucial al establecer los “valores naturales”. El precio natural de una mercancía, explica Dobb, es el punto de equilibrio hacia el cual tienden los precios de mercado en el largo plazo, determinado por las tasas naturales de salarios, beneficios y rentas. El precio de mercado puede desviarse temporalmente, pero siempre gravita en torno a este centro, funcionando como una ley reguladora que vincula las esferas de la producción y el intercambio.

Respecto al trabajo y el valor, \textcite{dobb1998} observa que Smith presentó una notable inconsistencia teórica. En su análisis del “temprano y rudo estado de la sociedad”, Smith formuló una teoría del valor-trabajo, donde la cantidad de trabajo necesaria para obtener un bien determinaba su valor. No obstante, al introducir el capital y la propiedad territorial, Smith abandonó esta formulación, sustituyéndola por la teoría de la “suma de los ingresos”, según la cual el valor se compone de salarios, beneficios y rentas. Este cambio —que Dobb considera una “regresión teórica”— debilitó la coherencia de la teoría clásica del valor.

Por otra parte, Smith buscó una medida del valor independiente de las fluctuaciones monetarias, proponiendo el trabajo economizado como patrón invariable, ya que el trabajo es lo único que “permanece constante en su propio valor”. \textcite{dobb1998} subraya que esta concepción, aunque ingeniosa, confunde el trabajo como medida con el trabajo como causa del valor. En la misma línea, el autor destaca la importancia de la división del trabajo como el motor del progreso económico, al incrementar la productividad y ampliar la capacidad de acumulación del sistema.

\section*{Smith frente a los Fisiócratas y la Crítica Ricardiana}

La relación de Smith con sus predecesores y sucesores, según \textcite{dobb1998}, revela tanto afinidades como divergencias.

Respecto a los fisiócratas, Dobb aclara que el vínculo entre Smith y la escuela francesa fue de paralelismo más que de dependencia directa. Ambos coincidieron en considerar la economía como un sistema gobernado por leyes objetivas y naturales. Sin embargo, mientras los fisiócratas atribuían el origen del excedente únicamente a la agricultura, Smith amplió esta concepción, reconociendo que todas las actividades productivas podían generar valor y riqueza. Dobb observa que Smith adoptó de ellos la noción del capital como “adelanto” o anticipo a la producción, pero rechazó la exclusividad del \textit{produit net agrícola}.

Por otra parte, \textcite{dobb1998} subraya que Smith redefinió el “ingreso neto” no como un simple excedente natural de la tierra, sino como el conjunto de los fondos destinados al consumo y la reproducción social. De esta manera, Smith sustituyó el marco agrario de los fisiócratas por un enfoque más amplio, basado en la acumulación de capital y el intercambio mercantil.

Respecto a David Ricardo, Dobb destaca que este último ofreció una crítica más rigurosa a las inconsistencias del pensamiento \textit{smithiano}, especialmente en torno al valor. Ricardo acusó a Smith de confundir el “trabajo incorporado” con el “trabajo comandado”, lo que llevó a errores analíticos en su teoría de los precios relativos. Dobb retoma esta crítica para mostrar que Smith oscilaba entre una visión objetiva del valor y una interpretación subjetiva del mismo, generando ambigüedad en la relación entre costos de producción y precios naturales.

Asimismo, \textcite{dobb1998} enfatiza que Smith, al enfocarse en el producto bruto en lugar del producto neto, perdió de vista el origen del excedente, una omisión que Ricardo y Marx intentaron corregir posteriormente.

\section*{La Interpretación de Dobb sobre Adam Smith}

Desde su perspectiva materialista, \textcite{dobb1998} valora en Smith su sentido histórico: el reconocimiento de que la acumulación de capital y la propiedad de la tierra introducen nuevas relaciones sociales en la producción. Dobb considera que en Smith existe una “teoría embrionaria de la deducción”, es decir, una comprensión inicial de que los beneficios y rentas constituyen deducciones del producto total del trabajo. Esta intuición, aunque no sistematizada, anticipa la teoría marxista de la plusvalía.

No obstante, \textcite{dobb1998} también señala las limitaciones del pensamiento \textit{smithiano}. El abandono de la teoría del valor-trabajo en favor de la “teoría de la suma” diluyó su rigor analítico. Asimismo, la ambigüedad en torno al concepto de trabajo productivo e improductivo es, según Dobb, un reflejo de la tensión entre la realidad histórica y la ideología liberal de la época. Smith osciló entre definir lo productivo como aquello que rinde un beneficio o como el trabajo que se incorpora en una mercancía tangible. Dobb insiste en que, desde una perspectiva capitalista, el trabajo productivo debe definirse como aquel que produce plusvalía para el capitalista.

Finalmente, \textcite{dobb1998} interpreta el ataque de Smith a los comerciantes y manufactureros no como una crítica al capitalismo en sí, sino como una defensa del proceso general de acumulación contra los privilegios mercantiles del sistema anterior. En este sentido, Smith no fue un opositor del capitalismo industrial, sino su apologista más coherente, pues su obra sirvió para legitimar el laissez-faire y la competencia irrestricta como principios del orden económico moderno.

\section*{Conclusión}

Adam Smith consolidó la economía política clásica al establecer el orden económico como un sistema gobernado por leyes naturales y regido por el interés individual. Sus aportes, según \textcite{dobb1998}, residen en haber identificado los mecanismos fundamentales del mercado, el papel central de la acumulación de capital y la necesidad de la competencia como fuerza reguladora.

Sin embargo, Dobb destaca que Smith se apartó de una teoría del valor basada en el trabajo, sustituyéndola por una teoría de los ingresos que oscurece el origen del excedente. A pesar de sus contradicciones, Smith reconoció el carácter histórico de la acumulación y la naturaleza social de la producción, aportando las bases para la crítica marxista de la economía política.

La interpretación de \textcite{dobb1998} conserva plena vigencia porque subraya que el análisis económico debe centrarse en las relaciones sociales de producción y en la distribución del excedente, más que en el simple intercambio de mercancías. De este modo, Dobb rescata de Smith tanto su legado científico como sus límites ideológicos, permitiendo comprender el tránsito del pensamiento económico desde el liberalismo clásico hacia la crítica marxista.

\printbibliography

\end{document}

%% 
%% Copyright (C) 2019 by Daniel A. Weiss <daniel.weiss.led at gmail.com>
%% 
%% This work may be distributed and/or modified under the
%% conditions of the LaTeX Project Public License (LPPL), either
%% version 1.3c of this license or (at your option) any later
%% version.  The latest version of this license is in the file:
%% 
%% http://www.latex-project.org/lppl.txt
%% 
%% Users may freely modify these files without permission, as long as the
%% copyright line and this statement are maintained intact.
%% 
%% This work is not endorsed by, affiliated with, or probably even known
%% by, the American Psychological Association.
%% 
%% This work is "maintained" (as per LPPL maintenance status) by
%% Daniel A. Weiss.
%% 
%% This work consists of the file  apa7.dtx
%% and the derived files           apa7.ins,
%%                                 apa7.cls,
%%                                 apa7.pdf,
%%                                 README,
%%                                 APA7american.txt,
%%                                 APA7british.txt,
%%                                 APA7dutch.txt,
%%                                 APA7english.txt,
%%                                 APA7german.txt,
%%                                 APA7ngerman.txt,
%%                                 APA7greek.txt,
%%                                 APA7czech.txt,
%%                                 APA7turkish.txt,
%%                                 APA7endfloat.cfg,
%%                                 Figure1.pdf,
%%                                 shortsample.tex,
%%                                 longsample.tex, and
%%                                 bibliography.bib.
%% 
%%
%%
%% This is file `./samples/shortsample.tex',
%% generated with the docstrip utility.
%%
%% The original source files were:
%%
%% apa7.dtx  (with options: `shortsample')
%% ----------------------------------------------------------------------
%% 
%% apa7 - A LaTeX class for formatting documents in compliance with the
%% American Psychological Association's Publication Manual, 7th edition
%% 
%% Copyright (C) 2019 by Daniel A. Weiss <daniel.weiss.led at gmail.com>
%% 
%% This work may be distributed and/or modified under the
%% conditions of the LaTeX Project Public License (LPPL), either
%% version 1.3c of this license or (at your option) any later
%% version.  The latest version of this license is in the file:
%% 
%% http://www.latex-project.org/lppl.txt
%% 
%% Users may freely modify these files without permission, as long as the
%% copyright line and this statement are maintained intact.
%% 
%% This work is not endorsed by, affiliated with, or probably even known
%% by, the American Psychological Association.
%% 
%% ----------------------------------------------------------------------