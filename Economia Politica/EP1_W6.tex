\documentclass[stu,12pt,floatsintext]{apa7}


\usepackage[american]{babel}
\usepackage{csquotes}
\usepackage[style=apa,sortcites=true,sorting=nyt,backend=biber]{biblatex}
\DeclareLanguageMapping{american}{american-apa} % APA (7.ª con biblatex-apa actual)
\addbibresource{Bib.bib} % Bibliography file


\DefineBibliographyStrings{american}{
  references = {Referencias},
}

\usepackage[T1]{fontenc} 
\usepackage{mathptmx} 

% Title page stuff _____________________
\title{Semana 6 - La Producción Mercantil} % The big, long version of the title for the title page
% \shorttitle{APA Starter} % The short title for the header
\author{David Armando Ramirez Navarrete}
\duedate{21 septiembre 2025}
% \date{January 17, 2024} The student version doesn't use the \date command, for whatever reason
\affiliation{Facultad de Economía (UNAM)}
\course{Economía Política I} % LaTeX gets annoyed (i.e., throws a grumble-error) if this is blank, so I put something here. However, if your instructor will mark you off for this being on the title page, you can leave this entry blank (delete the PSY 4321, but leave the command), and just make peace with the error that will happen. It won't break the document.
\professor{Gabriela Téllez Arriaga}  % Same situation as for the course info. Some instructors want this, some absolutely don't and will take off points. So do what you gotta.

% \abstract{This is the abstract for this paper, wherein the main points of the introduction, method, results, and discussion are quickly talked about. Probably in more than one sentence, though. Dare I guess, more than two? There is a page break before starting the Introduction.}

%\keywords{APA style, demonstration} % If you need to have keywords for your paper, delete the % at the start of this line

\begin{document}

\maketitle

\section*{Introducción}

Rosa Luxemburgo (1871-1919) es una de las pensadoras marxistas más relevantes. En Introducción a la economía política, afirma que "la producción mercantil es el primer estadio de la producción capitalista y su premisa histórica" (\cite{rosalux1975}). El análisis de este concepto en el Capítulo 4 es fundamental para entender cómo la forma mercantil prepara el terreno para las contradicciones del capitalismo.

\section*{¿Cómo funciona la producción mercantil?}

Luxemburgo señala que “la producción mercantil presupone la existencia de productores privados independientes, cada uno de los cuales dirige de manera autónoma su propio proceso de trabajo” (\cite{rosalux1975}). En este sistema, los productos no se destinan al consumo directo, sino al intercambio: “El producto del trabajo deja de ser un objeto de consumo inmediato para convertirse en mercancía destinada al mercado” (\cite{rosalux1975}).

Este mecanismo muestra que el mercado, más que un simple canal de distribución, actúa como regulador externo y coactivo de la actividad productiva.

\section*{¿Cómo conoce un individuo su valía en la producción mercantil?}

\textcite{rosalux1975} enfatiza que la división social del trabajo en la producción mercantil no es planificada conscientemente, sino que se regula \textit{ex post facto} a través del mercado.

Así, el sistema ajusta las proporciones entre sectores productivos mediante fluctuaciones de precios y competencia: “El movimiento de los precios es el que decide si una determinada rama de producción se ha desarrollado demasiado o demasiado poco” (\cite{rosalux1975}).

\section*{¿Cómo conoce un individuo su valía en la producción mercantil?}

Sobre la valoración del trabajo individual, Luxemburgo explica que “el productor privado sólo conoce la valía de su trabajo cuando logra vender su producto en el mercado” (\cite{rosalux1975}). 

Si el producto se vende por debajo del precio medio, el productor comprende que su trabajo no fue socialmente necesario, y viceversa. Esta mediación del mercado cosifica las relaciones sociales en la forma de valor.

\section{Señale cuáles son las contradicciones que la autora encuentra en la producción mercantil}

Luxemburgo identifica varias contradicciones estructurales que caracterizan a la producción mercantil:

\begin{itemize}
  \item \textbf{Producción privada vs. trabajo social.} La producción se organiza de manera privada e independiente, mientras que el trabajo tiene un carácter social, lo que genera tensiones entre decisiones individuales y necesidades colectivas (\cite{rosalux1975}).
  \item \textbf{Anarquía de la producción.} Al carecer de planificación consciente, la coordinación entre sectores ocurre solo a través de las fluctuaciones del mercado. Esta regulación ciega genera desajustes y tensiones permanentes (\cite{rosalux1975}).
  \item \textbf{Tendencia a la sobreproducción y las crisis.} Al desconocer el límite social de la demanda, los productores exceden con frecuencia las necesidades del mercado, lo que provoca crisis de sobreproducción: “La producción mercantil engendra inevitablemente crisis de sobreproducción” (\cite{rosalux1975}).
  \item \textbf{Desigualdades y polarización social.} El sistema favorece a los productores más eficientes y arruina a los menos competitivos, concentrando los medios de producción y profundizando las desigualdades: “En la producción mercantil reside ya en germen toda la contradicción del sistema capitalista” (\cite{rosalux1975}).
\end{itemize}

Estas contradicciones muestran que la producción mercantil es inestable y prepara las condiciones para el desarrollo pleno del capitalismo.

\section*{Conclusión}

El estudio de Luxemburgo demuestra que la producción mercantil es más que una forma simple de intercambio: es la base histórica del capitalismo y el germen de sus contradicciones. El mercado regula la división del trabajo y mide la valía individual, pero lo hace de forma anárquica, generando desigualdades y crisis. Como advierte Rosa Luxemburgo, “en la producción mercantil reside ya en germen toda la contradicción del sistema capitalista” (\cite{rosalux1975}).

Estas ideas mantienen plena vigencia para analizar las dinámicas económicas contemporáneas, donde las tensiones entre producción privada y necesidades sociales persisten bajo nuevas formas.

\printbibliography

\end{document}

%% 
%% Copyright (C) 2019 by Daniel A. Weiss <daniel.weiss.led at gmail.com>
%% 
%% This work may be distributed and/or modified under the
%% conditions of the LaTeX Project Public License (LPPL), either
%% version 1.3c of this license or (at your option) any later
%% version.  The latest version of this license is in the file:
%% 
%% http://www.latex-project.org/lppl.txt
%% 
%% Users may freely modify these files without permission, as long as the
%% copyright line and this statement are maintained intact.
%% 
%% This work is not endorsed by, affiliated with, or probably even known
%% by, the American Psychological Association.
%% 
%% This work is "maintained" (as per LPPL maintenance status) by
%% Daniel A. Weiss.
%% 
%% This work consists of the file  apa7.dtx
%% and the derived files           apa7.ins,
%%                                 apa7.cls,
%%                                 apa7.pdf,
%%                                 README,
%%                                 APA7american.txt,
%%                                 APA7british.txt,
%%                                 APA7dutch.txt,
%%                                 APA7english.txt,
%%                                 APA7german.txt,
%%                                 APA7ngerman.txt,
%%                                 APA7greek.txt,
%%                                 APA7czech.txt,
%%                                 APA7turkish.txt,
%%                                 APA7endfloat.cfg,
%%                                 Figure1.pdf,
%%                                 shortsample.tex,
%%                                 longsample.tex, and
%%                                 bibliography.bib.
%% 
%%
%%
%% This is file `./samples/shortsample.tex',
%% generated with the docstrip utility.
%%
%% The original source files were:
%%
%% apa7.dtx  (with options: `shortsample')
%% ----------------------------------------------------------------------
%% 
%% apa7 - A LaTeX class for formatting documents in compliance with the
%% American Psychological Association's Publication Manual, 7th edition
%% 
%% Copyright (C) 2019 by Daniel A. Weiss <daniel.weiss.led at gmail.com>
%% 
%% This work may be distributed and/or modified under the
%% conditions of the LaTeX Project Public License (LPPL), either
%% version 1.3c of this license or (at your option) any later
%% version.  The latest version of this license is in the file:
%% 
%% http://www.latex-project.org/lppl.txt
%% 
%% Users may freely modify these files without permission, as long as the
%% copyright line and this statement are maintained intact.
%% 
%% This work is not endorsed by, affiliated with, or probably even known
%% by, the American Psychological Association.
%% 
%% ----------------------------------------------------------------------