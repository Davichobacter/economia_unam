\documentclass[11pt, a4paper]{article}

% --- PAQUETES ---
\usepackage[a4paper, margin=1in]{geometry} % Márgenes estándar
\usepackage[T1]{fontenc}
\usepackage[utf8]{inputenc}
\usepackage[spanish]{babel}
\usepackage{helvet} % Tipografía sans-serif (Helvetica)
\renewcommand{\familydefault}{\sfdefault}
\usepackage{xcolor}
\usepackage{titlesec}
\usepackage{enumitem}

% --- CONFIGURACIÓN DE LISTAS ---
\setlist[itemize]{leftmargin=*, nosep}

% --- COLORES ---
\definecolor{primarycolor}{HTML}{003B6F} % Azul oscuro profesional
\definecolor{graytext}{HTML}{444444}

% --- FORMATO DE SECCIONES ---
% Línea horizontal debajo de los títulos de sección (también funciona con \section*)
\titleformat{\section}
  {\Large\bfseries\color{black}}
  {} % sin etiqueta/numero
  {0em}
  {}[\titlerule]
\titlespacing*{\section}{0pt}{12pt}{10pt}

% --- ENLACES (debe ir al final) ---
\usepackage{hyperref}
\hypersetup{
  colorlinks=true,
  linkcolor=primarycolor,
  urlcolor=primarycolor
}

% --- DOCUMENTO ---
\pagestyle{empty} % Sin números de página

\begin{document}

% --- ENCABEZADO: NOMBRE Y CONTACTO ---
\begin{center}
    {\Huge \bfseries David Armando Ramírez Navarrete}\\[8pt]
    \color{graytext}
    Ciudad de México, México \\
    navarreteram.david@gmail.com \textbar{} +52 747 102 2550
\end{center}

\vspace{8pt}

% --- PERFIL PROFESIONAL ---
\section*{Perfil Profesional}
Ingeniero en Sistemas Computacionales con experiencia en desarrollo de aplicaciones web y entornos mainframe. Apasionado por la programación, con enfoque en soluciones eficientes e innovadoras. Comprometido con el aprendizaje continuo y la mejora de habilidades técnicas, especialmente en desarrollo de software. Adaptable, proactivo y con alta capacidad para trabajar en equipo y adoptar nuevas tecnologías.

% --- EXPERIENCIA PROFESIONAL ---
\section*{Experiencia Profesional}

\textbf{NTT DATA - BBVA} \hfill \textit{Julio 2025 -- Actualidad} \\
\textit{Becario COBOL}
\begin{itemize}
  \item Revisión de documentación por cambios productivos post instalación y validación de componentes.
  \item Creación de tablas dinámicas y reportes de productividad con Google Looker Studio.
  \item Administración y control de periodos vacacionales y ejecución de despliegues en desarrollo.
  \item \textbf{Herramientas Clave:} Google Looker Studio, Google App Script, Jira, Google Sheets.
\end{itemize}

\vspace{10pt}

\textbf{NTT Data} \hfill \textit{Marzo 2025 -- Julio 2025} \\
\textit{Programa de Formación COBOL}
\begin{itemize}
  \item Desarrollo y mantenimiento de programas COBOL para procesos bancarios.
  \item Optimización de consultas SQL en entornos mainframe.
  \item Implementación de soluciones en JCL y DB2 y colaboración en equipos Agile.
\end{itemize}

\vspace{10pt}

\textbf{Google Cloud Platform (GCP) - Proyecto Personal} \hfill \textit{Marzo 2024 -- Marzo 2025} \\
\textit{Desarrollador Cloud}
\begin{itemize}
  \item Configuración de entornos, gestión de máquinas virtuales y automatización de tareas en GCP.
  \item Documentación y monitoreo de sistemas para la entrega de soluciones escalables.
\end{itemize}

% --- EDUCACIÓN ---
\section*{Educación}
\textbf{Universidad Virtual del Estado de Guanajuato} \hfill \textit{} \\
\textit{Ingeniería en Sistemas Computacionales}

% --- HABILIDADES TÉCNICAS Y BLANDAS ---
\section*{Habilidades}
\begin{itemize}
    \item \textbf{Lenguajes de Programación:} Java, JavaScript, Python, COBOL
    \item \textbf{Bases de Datos y Consultas:} SQL, QMF, DB2
    \item \textbf{Control de Versiones:} ChangeMan, Git, Bitbucket
    \item \textbf{Metodologías:} Marco de Trabajo Agile, Waterfall
    \item \textbf{Plataformas y Otros:} Google Cloud Platform, TSO, ISPF, JCL
    \item \textbf{Habilidades Blandas:} Trabajo en Equipo, Adaptabilidad, Resolución de Problemas, Autodidacta.
\end{itemize}

% --- CERTIFICACIONES Y CURSOS ---
\section*{Certificaciones y Cursos}
\begin{itemize}
  \item Certificación HOST Standard (Desarrollo Seguro)
  \item Scrum Master
  \item Google Cloud Computing Foundations (Google)
  \item Oracle Next Education -- G3 (Alura Latam)
\end{itemize}

% --- IDIOMAS ---
\section*{Idiomas}
\begin{itemize}
    \item \textbf{Español:} Nativo
    \item \textbf{Inglés:} Nivel B1 (Intermedio)
\end{itemize}

\end{document}
