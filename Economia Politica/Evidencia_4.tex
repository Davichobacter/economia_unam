\documentclass[12pt]{article}

% Configuración básica
\usepackage[spanish]{babel}
\usepackage[utf8]{inputenc}
\usepackage[T1]{fontenc}
\usepackage{geometry}
\usepackage{setspace}
\usepackage{hyperref}
\usepackage{enumitem}
\usepackage{graphicx}
\usepackage[section]{placeins} 

\geometry{letterpaper, margin=2.5cm}
\onehalfspacing

% Datos del informe (puedes modificarlos)
\title{Avance 2 \\ Reporte de Depuración y Evaluación de Componentes de Software}
\author{David Armando Ramírez Navarrete}
\date{11 de enero de 2026}

\begin{document}

\maketitle

\section*{Resumen}
El presente informe técnico documenta los hallazgos obtenidos durante la revisión inicial de la documentación de Looker Studio, como parte del análisis de información realizado en las primeras semanas de prácticas. El trabajo refleja habilidades de análisis, revisión estructurada de documentación técnica y atención al detalle.

\section*{Introducción}
Las prácticas profesionales representan una oportunidad para aplicar conocimientos académicos en un entorno real de trabajo. Durante las primeras semanas se desarrolló un proceso de análisis de documentación técnica con el fin de comprender herramientas, flujos de datos y procesos utilizados en la organización.

Este informe presenta los hallazgos obtenidos a partir de la revisión de la documentación oficial de Looker Studio, con énfasis en la comprensión de la plataforma y en la identificación de oportunidades de mejora o aprendizaje relevante para el proyecto asignado.

\section{Descripción de la actividad}
Contexto de la revisión y tipo de sistemas o componentes analizados
(sin mencionar nombres ni plataformas).

\section{Metodología de revisión}
Criterios utilizados para evaluar el código:
- estructura
- validaciones
- manejo de errores
- legibilidad
- lógica general

\section{Hallazgos}
Listado de errores, inconsistencias o áreas de mejora detectadas.

\section{Acciones realizadas o sugeridas}
Correcciones aplicadas o recomendaciones propuestas.

\section{Resultados}
Beneficios obtenidos a partir de la revisión realizada.

\section{Conclusiones}
Aprendizajes técnicos y competencias desarrolladas.

\section*{Referencias}

\begin{itemize}
    \item Google Cloud. (n.d.). \textit{Cómo crear campos calculados en Looker Studio}.  
    Recuperado de: \url{https://docs.cloud.google.com/looker/docs/studio/tutorial-create-calculated-fields-in-looker-studio?hl=es-419}

    \item Google Cloud. (n.d.). \textit{Variables de resultado de consulta en Looker Studio}.  
    Recuperado de: \url{https://docs.cloud.google.com/looker/docs/studio/query-result-variables?hl=es-419}

    \item Google Cloud. (n.d.). \textit{Agregar, editar y solucionar problemas con campos calculados en Looker Studio}.  
    Recuperado de: \url{https://docs.cloud.google.com/looker/docs/studio/add-edit-and-troubleshoot-calculated-fields?hl=es-419}
\end{itemize}

\end{document}

